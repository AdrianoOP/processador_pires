\documentclass[../main.tex]{subfiles} 

\begin{document}

%%%%%%%%%%%%instrucao ADD
\subsection{ADD}

\begin{table}[H]
	\centering
	\begin{tabular}{|c|c|c|c|c|c|} %6 colunas
	\hline
	CLK & PC & IR & A& B & Sx \\\hline
	3 & 0b00000001 & 0bZZZZZZZZ & 0bXXXXXXXX & 0bYYYYYYYY & S2 \\\hline 
	4 & 0b00000001 & 0bZZZZZZZZ & A+B & 0bYYYYYYYY & S3 \\\hline
	5 & 0b00000001 & 0bZZZZZZZZ & A+B & 0bYYYYYYYY & S4 \\\hline
	6 & 0b00000002 & 0bZZZZZZZZ & A+B & 0bYYYYYYYY & S0 \\\hline	
	\end{tabular}
\end{table}

%%%%%%%%%%%%instrucao MUL
\subsection{MUL}

\begin{table}[H]
	\centering
	\begin{tabular}{|c|c|c|c|c|c|} %6 colunas
	\hline
	CLK & PC & IR & A& B & Sx  \\\hline
	3 & 0b00000001 & 0bZZZZZZZZ & 0bXXXXXXXX & 0bYYYYYYYY & S2 \\\hline 
	4 & 0b00000001 & 0bZZZZZZZZ & LSB(A*B) & 0bYYYYYYYY & S7 \\\hline
	5 & 0b00000001 & 0bZZZZZZZZ & LSB(A*B) & HSB(A*B) & S8 \\\hline
	6 & 0b00000002 & 0bZZZZZZZZ & LSB(A*B) & HSB(A*B) & S0 \\\hline	
	\end{tabular}
\end{table}

%%%%%%%%%%%%instrucao AND
\subsection{AND}

\begin{table}[H]
	\centering
	\begin{tabular}{|c|c|c|c|c|c|} %6 colunas
	\hline
	CLK & PC & IR & A& B & Sx \\\hline
	3 & 0b00000001 & 0bZZZZZZZZ & 0bXXXXXXXX & 0bYYYYYYYY & S2 \\\hline 
	4 & 0b00000001 & 0bZZZZZZZZ & A AND B & 0bYYYYYYYY & S3 \\\hline
	5 & 0b00000001 & 0bZZZZZZZZ & A AND B & 0bYYYYYYYY & S4 \\\hline
	6 & 0b00000002 & 0bZZZZZZZZ & A AND B & 0bYYYYYYYY & S0 \\\hline	
	\end{tabular}
\end{table}


%%%%%%%%%%%%instrucao OR
\subsection{OR}

\begin{table}[H]
	\centering
	\begin{tabular}{|c|c|c|c|c|c|} %6 colunas
	\hline
	CLK & PC & IR & A& B & Sx \\\hline
	3 & 0b00000001 & 0bZZZZZZZZ & 0bXXXXXXXX & 0bYYYYYYYY & S2 \\\hline 
	4 & 0b00000001 & 0bZZZZZZZZ & A OR B & 0bYYYYYYYY & S3 \\\hline
	5 & 0b00000001 & 0bZZZZZZZZ & A OR B & 0bYYYYYYYY & S4 \\\hline
	6 & 0b00000002 & 0bZZZZZZZZ & A OR B & 0bYYYYYYYY & S0 \\\hline	
	\end{tabular}
\end{table}

%%%%%%%%%%%%instrucao NOT
\subsection{NOT}

\begin{table}[H]
	\centering
	\begin{tabular}{|c|c|c|c|c|c|} %6 colunas
	\hline
	CLK & PC & IR & A& B & Sx \\\hline
	3 & 0b00000001 & 0bZZZZZZZZ & 0bXXXXXXXX & 0bYYYYYYYY & S2 \\\hline 
	4 & 0b00000001 & 0bZZZZZZZZ & NOT A & 0bYYYYYYYY & S3 \\\hline
	5 & 0b00000001 & 0bZZZZZZZZ & NOT A & 0bYYYYYYYY & S4 \\\hline
	6 & 0b00000002 & 0bZZZZZZZZ & NOT A & 0bYYYYYYYY & S0 \\\hline	
	\end{tabular}
\end{table}

%%%%%%%%%%%%instrucao SWP
\subsection{SWP}

\begin{table}[H]
	\centering
	\begin{tabular}{|c|c|c|c|c|c|c|} %7 colunas
	\hline
	CLK & PC & IR & A& B & Sx & Observação\\\hline
	3 & 0b00000001 & 0bZZZZZZZZ & 0bXXXXXXXX & 0bYYYYYYYY & S2 & - \\\hline 
	4 & 0b00000001 & 0bZZZZZZZZ & B & 0bYYYYYYYY & S3 & $ULA\_en=0$\\\hline
	5 & 0b00000001 & 0bZZZZZZZZ & B & A & S5 & $ULA\_en=1$\\\hline
	6 & 0b00000001 & 0bZZZZZZZZ & B & A & S4 & - \\\hline
	7 & 0b00000002 & 0bZZZZZZZZ & B & A & S0 & - \\\hline	
	\end{tabular}
\end{table}

No estado S3, o processador joga o valor de A para a ULA e depois a desabilita (comportamento de latch). Como A só carrega o seu valor na borda
de descida do clock, é possível no mesmo ciclo Armazenar B em A e deixar o valor antigo de A armazenado (através de Latches) na ULA.

%%%%%%%%%%%%instrucao LI
\subsection{LI const}

\begin{table}[H]
	\centering
	\begin{tabular}{|c|c|c|c|c|c|} %6 colunas
	\hline
	CLK & PC & IR & A& B & Sx \\\hline
	3 & 0b00000001 & const & 0bXXXXXXXX & 0bYYYYYYYY & S2 \\\hline 
	4 & 0b00000001 & const & const & 0bYYYYYYYY & S3 \\\hline
	5 & 0b00000001 & const & const & 0bYYYYYYYY & S4 \\\hline
	6 & 0b00000002 & const & const & 0bYYYYYYYY & S0 \\\hline	
	\end{tabular}
\end{table}

\section{Instruções tipo I}

%%%%%%%%%%%%instrucao LI
\subsection{HALT}
\begin{table}[H]
	\centering
	\begin{tabular}{|c|c|c|c|c|c|} %6 colunas
	\hline
	CLK & PC & IR & A& B & Sx \\\hline
	3 & 0b00000001 & 0bZZZZZZZZ & 0bXXXXXXXX & 0bYYYYYYYY & S2 \\\hline 
	6 & 0b00000000 & 0bZZZZZZZZ & 0bXXXXXXXX & 0bYYYYYYYY & S0 \\\hline	
	\end{tabular}
\end{table}

\end{document}