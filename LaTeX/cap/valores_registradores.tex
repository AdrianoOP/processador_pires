\documentclass[../main.tex]{subfiles} 
\begin{document}
Para as análises em questão será considerado, para cada função, que todos os registradores estão zerados e o bloco controlador está em seu estado inicial.

Todas as instruções terão em comum o mesmo comportamento dos registradores de acordo com a tabela~\ref{tab:tabela_estados_iniciais} abaixo.

\begin{table}[H]
	\centering
	\caption{Tabela de valores de registradores comum a todos as instruções}
	\begin{tabular}{|c|c|c|c|c|c|} %7 colunas
	\hline
	CLK & PC & IR & A& B & Sx \\\hline
	1 & 0b00000000 & 0b00000000 & 0bXXXXXXXX & 0bYYYYYYYY & S0 \\\hline % estado inicial
	2 & 0b00000000 & instrução & 0bXXXXXXXX & 0bYYYYYYYY & S1 \\\hline % estado um
	\end{tabular}
	\label{tab:tabela_estados_iniciais}
\end{table}

Os demais comportamentos dos registradores para cada estado de todas as instruções estão descritos nas tabelas abaixo

		\section{Instruções tipo I}
			\subfile{cap/valores_registradores_tipoI}
		\section{Instruções tipo J}
			\subfile{cap/valores_registradores_tipoJ}
		\section{Instruções tipo R}
			\subfile{cap/valores_registradores_tipoR}
	
\end{document}