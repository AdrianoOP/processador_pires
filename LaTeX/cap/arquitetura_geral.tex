\documentclass[../main.tex]{subfiles} 
\begin{document}
	Para fins desta implementação, os conceitos utiliazados foram
	definidos a partir do diagrama da figura~\ref{fig:processador_pires}.
	
	\begin{figure}[H]
		\centering
		\includegraphics[height=0.6\textheight]{img/processador_pires}
		\caption{Diagrama geral do processador Pires}
		\label{fig:processador_pires}
	\end{figure}
	
	Para a arquitetura projetada um clock global sincronizará todo o hardware, composto dos seguintes elementos:
	\begin{itemize}
		\item Uma memória ROM, que irá guardar OPcodes (Já convertidos de assembly para binário) que são inseridos manualmente na ROM através de um método descrito na sessão~\ref{sec:ROM};
		\item Uma memória RAM:
			\begin{itemize}
				\item Com endereçamento para até 256 palavras de 8 bits; 
				\item Capaz de ler e escrever de registradores especiais do Bloco de dados; 
				\item Com controle de gravação realizado pelo Bloco de Controle;
				\item Com mapeamento de memória possibilitando conectar diretamente \textit{Inputs} e \textit{Outputs} do kit de desenvolvimento;
			\end{itemize}
		\item Um Bloco de Dados que armazena os dados temporários do processador (através dos registradores IR, PC, A e B) e que é responsável pelo processamento lógico aritmético;
		\item Um Bloco de Controle que, dependendo da instrução, dos Flags da ULA e de uma máquina de estados, fornece os sinais que coordenam o funcionamento dos demais hardwares já descritos;
	\end{itemize}
\end{document}