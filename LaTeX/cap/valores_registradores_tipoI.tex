\documentclass[../main.tex]{subfiles} 

\begin{document}

\subsection{LDA end}

\begin{table}[H]
	\centering
	\begin{tabular}{|c|c|c|c|c|c|} %6 colunas
	\hline
	CLK & PC & IR & A& B & Sx \\\hline
	3 & 0b00000001 & end & 0bXXXXXXXX & 0bYYYYYYYY & S2 \\\hline 
	4 & 0b00000001 & end & valor & 0bYYYYYYYY & S3 \\\hline
	5 & 0b00000001 & end & valor & 0bYYYYYYYY & S4 \\\hline
	6 & 0b00000002 & end & valor & 0bYYYYYYYY & S0 \\\hline	
	\end{tabular}
\end{table}


%%%%%%%%%%%%instrucao STA
\subsection{STA end}

\begin{table}[H]
	\centering
	\begin{tabular}{|c|c|c|c|c|c|} %6 colunas
	\hline
	CLK & PC & IR & A& B & Sx \\\hline
	3 & 0b00000001 & end & valor & 0bYYYYYYYY & S2 \\\hline 
	4 & 0b00000001 & end & valor & 0bYYYYYYYY & S3 \\\hline
	5 & 0b00000001 & end & valor & 0bYYYYYYYY & S4 \\\hline
	6 & 0b00000002 & end & valor & 0bYYYYYYYY & S0 \\\hline		
	\end{tabular}
\end{table}

STA grava na RAM no estado S3


%%%%%%%%%%%%instrucao BLT
\subsection{BLT end}

\begin{table}[H]
	\centering
	\begin{tabular}{|c|c|c|c|c|c|} %6 colunas
	\hline
	CLK & PC & IR & A& B & Sx \\\hline
	3 & 0b00000001 & end & 0bXXXXXXXX & 0bYYYYYYYY & S2 \\\hline 
	4 & 0b00000001 & end & 0bXXXXXXXX & 0bYYYYYYYY & S3 \\\hline 	
	\end{tabular}	
\end{table}

Se A $<$ B, a ULA gera o \textit{flag \textbf{N}} e o comportamento será:

\begin{table}[H]
	\centering
	\begin{tabular}{|c|c|c|c|c|c|} %6 colunas
	\hline
	CLK & PC & IR & A& B & Sx \\\hline
	5 & 0b00000001 & end & 0bXXXXXXXX & 0bYYYYYYYY & S6 \\\hline 
	6 & end & end & 0bXXXXXXXX & 0bYYYYYYYY & S0 \\\hline 	
	\end{tabular}	
\end{table}

Se A $>$ B, a ULA não gera o \textit{flag \textbf{N}} e o comportamento será:

\begin{table}[H]
	\centering
	\begin{tabular}{|c|c|c|c|c|c|} %6 colunas
	\hline
	CLK & PC & IR & A& B & Sx \\\hline
	5 & 0b00000001 & end & 0bXXXXXXXX & 0bYYYYYYYY & S6 \\\hline 
	6 & 0b00000002 & end & 0bXXXXXXXX & 0bYYYYYYYY & S0 \\\hline 	
	\end{tabular}	
\end{table}

%%%%%%%%%%%%instrucao BEQ
\subsection{BEQ end}

\begin{table}[H]
	\centering
	\begin{tabular}{|c|c|c|c|c|c|} %6 colunas
	\hline
	CLK & PC & IR & A& B & Sx \\\hline
	3 & 0b00000001 & end & 0bXXXXXXXX & 0bYYYYYYYY & S2 \\\hline 
	4 & 0b00000001 & end & 0bXXXXXXXX & 0bYYYYYYYY & S3 \\\hline 	
	\end{tabular}	
\end{table}

Se A $=$ B, a ULA gera o \textit{flag \textbf{Z}} e o comportamento será:

\begin{table}[H]
	\centering
	\begin{tabular}{|c|c|c|c|c|c|} %6 colunas
	\hline
	CLK & PC & IR & A& B & Sx \\\hline
	5 & 0b00000001 & end & 0bXXXXXXXX & 0bYYYYYYYY & S6 \\\hline 
	6 & end & end & 0bXXXXXXXX & 0bYYYYYYYY & S0 \\\hline 	
	\end{tabular}	
\end{table}

Se A $!=$ B, a ULA não gera o \textit{flag \textbf{Z}} e o comportamento será:

\begin{table}[H]
	\centering
	\begin{tabular}{|c|c|c|c|c|c|} %6 colunas
	\hline
	CLK & PC & IR & A& B & Sx \\\hline
	5 & 0b00000001 & end & 0bXXXXXXXX & 0bYYYYYYYY & S6 \\\hline 
	6 & 0b00000002 & end & 0bXXXXXXXX & 0bYYYYYYYY & S0 \\\hline 	
	\end{tabular}	
\end{table}

\end{document}