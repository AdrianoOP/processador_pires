\documentclass[a4paper]{ufscThesis}
\usepackage{url}
\usepackage[pdftex]{graphicx}
\usepackage[utf8]{inputenc}
\usepackage{float}
\usepackage{amsmath}
\usepackage{amsfonts}
\usepackage{amssymb}
\usepackage{subfiles}
\usepackage{listings}
\usepackage{color}


\definecolor{mygreen}{rgb}{0,0.6,0}
\definecolor{mygray}{rgb}{0.5,0.5,0.5}
\definecolor{mymauve}{rgb}{0.58,0,0.82}

\titulo{Projeto e implementação em VHDL para FPGAs de arquitetura com conjunto super-reduzido de instruções}
\subtitulo{Processador PIRES}
\autor{Adriano Oliveira Pires}
\data{10}{Setembro}{2013}




\orientador{Prof. Eduardo Augusto Bezerra}                    % Nome do orientador e (opcional) seu título
\coorientador{Prof. Djones Vinicius Lettnin}                % Nome do coorientador e seu título (opcional)
\coordenador{Prof. Chefe, Dr. Eng.}              % Nome do coordenador do curso e (opcional) seu título

\departamento[a]{Centro de Tecnologia e Ciência} %parametro a indica que eh feminino
\curso[a]{EEL 4100 38 - Sistemas Digitais e Dispositivos Lógicos Reconfiguráveis}%

%%% Sobre a Banca
\numerodemembrosnabanca{0} % Isso decide se haverá uma folha adicional
\orientadornabanca{não} % Se faz parte da banca definir como sim
\coorientadornabanca{não} % Se faz parte da banca definir como sim


\begin{document}
\def\PC{\textbf{PC}}
\def\A{\textbf{A}}
\def\B{\textbf{B}}
\def\IR{\textbf{IR}}

\capa
\folhaderosto{Trabalho de conclusão apresentado à disciplina EEL4100 38 - Sistemas Digitais e Dispositivos Lógicos Reconfiguráveis, para fim de avaliação parcial.} %opcao [comficha]

\chapter{Especificações}
	\subfile{cap/especificacoes}
\chapter{Arquitetura Projetada}
		\subfile{cap/arquitetura_geral}
	\section{Bloco de Dados}
		\subfile{cap/bloco_dados}
	\section{Bloco de controle}
		\subfile{cap/bloco_controle}
	\section{Valores dos Registradores para cada opcode}
		\subfile{cap/valores_registradores}
\chapter{Implementação}
	\section{Bloco de Dados}
		\subfile{cap/bloco_dados_real}
	
	\section{O Processador Pires}
	\label{sec:processador_real}
		\subfile{cap/processador_real}
	
	\section{Simulação de um programa em Assembly}
	\label{sec:simulacao}
		\subfile{cap/simulacao}
\end{document}
